\chapter{Progettazione Fisica}
\section{Creazione delle Tabelle}
\begin{lstlisting}
drop schema mtl cascade;
create schema mtl;

--create table mtl.author
create table mtl.author
(
    CodAuthor serial
        primary key,
    FName     varchar(20),
    LName     varchar(20)
);

-- create table mtl.series
create table mtl.series
(
    ISSN_S  issn primary key,
    Curator names,
    Edition int,
    Code_S  varchar(10),
    Name_S  names
);
--create table mtl.magazine
create table mtl.magazine
(
    ISSN_M            issn primary key,
    Name_M            names,
    Argument          names,
    Manager           names,
    YearRelease       timestamp,
    PublicationPeriod names,
    AccessMode        access,
    PublishingHouse   names
);


--create table mtl.book
create table mtl.book
(
    Doi_B            doi
        primary key,
    ISBN_B           varchar(13)
        unique,
    PublishingHouse  names,
    Language         names,
    AccessMode       access,
    Title            varchar(30),
    Argument         names,
    Reprint          boolean,
    Edition          int,
    ReleaseDate      timestamp,
    ReleaseLocation  location,
    PresentationName names,
    FK_Author        serial,
    FK_Series        issn,

    constraint BookFK_2 foreign key (Fk_Author) references mtl.Author (CodAuthor) on delete cascade,
    constraint BookFK_3 foreign key (FK_Series) references mtl.Series (ISSN_S) on delete set null
);

--create table mtl.article
create table mtl.article
(
    Doi_A           doi
        primary key,
    Title           varchar(40),
    AccessMode      access,
    YearRelease     timestamp,
    Editor          names,
    Topic           names,
    ReleaseDate     timestamp,
    ReleaseLocation location,
    ConferenceName  varchar(50),
    FK_Author       serial,
    FK_Magazine     issn,

    constraint ArticleFK_1 foreign key (FK_Author) references mtl.Author (CodAuthor) on delete cascade,
    constraint ArticleFK_2 foreign key (FK_Magazine) references mtl.Magazine (ISSN_M) on delete set null
);
\end{lstlisting}
\section{Creazione dei domini}
\begin{lstlisting}
	create domain issn as varchar(9)
check ( value like '%-%' );

create domain isbn as varchar(17)
check ( value like '%-_-%-%-_');

create domain doi as varchar(30)
check ( value like'10.%/%');

create domain access as varchar(20)
check ( value <> '' and value not similar to '%[0-9]+%'
        and value not similar to '%[@!#$ˆ∗%&]+%');

create domain names as varchar(30)
check ( value not similar to '%[@!#$ˆ∗%&]+%');

create domain location as varchar(50)
check ( value like '%,[0-9],%,[0-9],%');
\end{lstlisting}

\section{Creazione delle viste}
\begin{lstlisting}
create view mtl.bibliography as
select distinct b.Title,b.ReleaseDate,a.lname
from mtl.book b join mtl.author a on b.fk_author = a.codauthor
order by b.releasedate desc;

create view mtl.history as
select distinct a.fname, a.lname, ar.title,ar.yearrelease,ar.editor
from mtl.author a join mtl.article ar on a.codauthor = ar.fk_author
order by ar.yearrelease desc;

create view mtl.digital_goods as
select distinct b.title from mtl.book b where accessmode = 'Digital'
union
select distinct a.title from mtl.article a where accessmode = 'Digital'
union
select distinct m.name_m  from mtl.magazine m where accessmode = 'Digital'
union
select distinct s.name_s from mtl.series s join mtl.book b on s.issn_s = b.fk_series where b.accessmode='Digital';

create view mtl.paper_goods as
select distinct b.title from mtl.book b where accessmode = 'Paper'
union
select distinct a.title from mtl.article a where accessmode = 'Paper'
union
select distinct m.name_m  from mtl.magazine m where accessmode = 'Paper'
union
select distinct s.name_s from mtl.series s join mtl.book b on s.issn_s = b.fk_series where b.accessmode='Digital';

create view mtl.audio_goods as
select distinct b.title from mtl.book b where accessmode = 'Audio'
union
select distinct a.title from mtl.article a where accessmode = 'Audio'
union
select distinct m.name_m  from mtl.magazine m where accessmode = 'Audio'
union
select distinct s.name_s from mtl.series s join mtl.book b on s.issn_s = b.fk_series where b.accessmode='Digital';

create view mtl.presentation as
select b.title,a.fname,a.lname, b.presentationname, b.releaselocation, b.releasedate
from mtl.book b join mtl.author a on a.codauthor = b.fk_author;

create view mtl.discussion as
select ar.title,a.fname,a.lname, ar.conferencename, ar.releaselocation, ar.releasedate
from mtl.article ar join mtl.author a on a.codauthor = ar.fk_author
order by a.lname;
\end{lstlisting}

\section{Creazione di funzioni e trigger}
\begin{lstlisting}
	create or replace function mtl.function_1() returns trigger as
$$
declare
    stringa_in   varchar(13) = new.isbn_b;
    sum          integer     := 0;
    var_appoggio integer;
    resto        integer;
begin
    stringa_in := replace(stringa_in, '-', '');
    for i in 1..13
        loop
            var_appoggio = cast(substring(stringa_in from i for 1) as int);
            if (i % 2 = 0) then
                sum := sum + var_appoggio * 3;
            else
                sum := sum + var_appoggio;
            end if;
        end loop;
    resto = sum % 10;
    if (resto != 0) then
        delete from mtl.book where doi_b = new.doi_b;
    end if;
    return new;
end
$$
    language plpgsql;

create trigger validity_isbn
    after insert
    on mtl.book
    for each row
execute procedure mtl.function_1();

create or replace function mtl.function_2() returns trigger as
$$
declare
    stringa_in   varchar(13) = new.issn_s;
    sum          integer     := 0;
    var_appoggio integer;
    resto        integer;
begin
    stringa_in := replace(stringa_in, '-', '');
    for i in 1..8
        loop
            if substr(stringa_in, 8, 1) = 'X' then
                sum = sum + 10;
            end if;
            var_appoggio = cast(substring(stringa_in from i for 1) as int);
            if (i = 8) then
                sum = sum + 0;
            else
                sum := sum + var_appoggio * (9 - i);
            end if;
        end loop;
    resto = sum % 11;
    if (resto != 0) then
        delete from mtl.series where issn_s = new.issn_s;
    end if;
    return new;
end
$$
    language plpgsql;

create trigger validity_issn_s
    after insert
    on mtl.series
    for each row
execute procedure mtl.function_2();

create or replace function mtl.function_3() returns trigger as
$$
declare
    stringa_in   varchar(13) = new.issn_m;
    sum          integer     := 0;
    var_appoggio integer;
    resto        integer;
begin
    stringa_in := replace(stringa_in, '-', '');
    for i in 1..8
        loop
            if substr(stringa_in, 8, 1) = 'X' then
                sum = sum + 10;
            end if;
            var_appoggio = cast(substring(stringa_in from i for 1) as int);
            if (i = 8) then
                sum = sum + 0;
            else
                sum := sum + var_appoggio * (9 - i);
            end if;
        end loop;
    resto = sum % 11;
    if (resto != 0) then
        delete from mtl.magazine where issn_m = new.issn_m;
    end if;
    return new;
end
$$
    language plpgsql;

create trigger validity_issn_m
    after insert
    on mtl.magazine
    for each row
execute procedure mtl.function_3();
\end{lstlisting}